\documentclass[a4paper,10pt,twoside]{article}

\usepackage[inner=3cm,top=3cm,outer=2cm,bottom=3cm]{geometry}
\usepackage[T1]{fontenc}
\usepackage{amssymb}
\usepackage{fancyhdr}
\usepackage{fancyvrb}
\usepackage{graphicx}
\usepackage{xcolor}
\usepackage{listings}
\ifxetex
	% i xetex behövs inga fulhack för åäö
\else
	% åäö-hax för icke-xelatex
	\usepackage[utf8]{inputenc}
	% Fixar så man kan ha åäö i kodkommentarer
	\lstset{literate={ö}{{\"o}}1
		{ä}{{\"a}}1
		{å}{{\aa}}1
		{Ö}{{\"O}}1
		{Ä}{{\"A}}1
		{Å}{{\AA}}1
	}
\fi

\definecolor{dark-blue}{rgb}{0, 0, 0.6}
\definecolor{javared}{rgb}{0.6,0,0} % for strings
\definecolor{javagreen}{rgb}{0.25,0.5,0.35} % comments
\definecolor{javapurple}{rgb}{0.5,0,0.35} % keywords
\definecolor{javadocblue}{rgb}{0.25,0.35,0.75} % javadoc

\lstset{language=Java,
	basicstyle=\ttfamily,
	% Här anges alltså vilka färger som sak användas.
	% Vill ni inte ha färger, kommentera ut nästföljande fyra rader.
	keywordstyle=\color{javapurple}\bfseries,
	stringstyle=\color{javared},
	commentstyle=\color{javagreen},
	morecomment=[s][\color{javadocblue}]{/**}{*/},
	numbers=left, % Fixar radnumrering i vänstermarginalen
	numberstyle=\footnotesize,
	title=\lstname,
	showstringspaces=false,
	fancyvrb=true,
	extendedchars=true,
	breaklines=true,
	breakatwhitespace=true,
	tabsize=4 %Indenteringsstorlek
}

% Header och footer
\pagestyle{fancy}\headheight 13pt
%\lhead{\course\ -\ Inlämning \homeworknumber}
%\rhead{\theauthor\ -\ \thedate}
\fancyfoot{}
\fancyfoot[LE,RO]{\thepage}
\date{\thedate}
\author{\theauthor}
\renewcommand{\headrulewidth}{0pt}
\renewcommand{\footrulewidth}{0pt}

\def\theauthor{Kristoffer Emanuelsson} % TODO: stoppa in ditt namn här
\def\homeworknumber{4} % TODO: stoppa in vilken hemläxa det är här
\def\coursename{Introduktion till datalogi} % TODO: stoppan in kursnamn här
\def\course{DD1341} % TODO: stoppa in kurskod här
\def\coursegroup{5} % TODO: stoppa in gruppnummer
\def\courseleader{Fredrik Stockman} % TODO: stoppa in ovningsledare

\def\thedate{\today} % Byt ut om du vill ha annat datum på inlämningen

% Här börjar inlämningsspecifikt
\def\homeworknumber{HT1} % TODO: stoppa in vilken hemläxa det är här
\def\thedate{\today} % Byt ut om du vill ha annat datum på inlämningen

\title{Inlämning \homeworknumber\ - \course\ \coursename}

\begin{document}

% Labbcover + clearpage för blank baksida bakom den
%\textheight=24.0cm
%\textwidth=15.1cm
%
%\topmargin=-1.5cm
%%\headheight=0cm
%\headsep=0.7cm
%\oddsidemargin=0.4cm
%\evensidemargin=0.4cm
%%\columnsep=0.8cm
%\footskip=1.5cm
%\setcounter{secnumdepth}{0}
%\pagestyle{empty}
%%\markboth{}{Namn: \hfill Personnr: \hfill}
%%\markboth{DD1341 inda11}{DD1341 inda11}

%\begin{document}

\thispagestyle{empty}
\section*{DD1339 Introduktion till datalogi}
%\section*{DD1339 Introduktion till datalogi 2012/2013}

\vspace{10mm}

\subsection*{Uppgift nummer: ......}

\vspace{3mm}

\subsection*{Namn: ...........................................................................}

\vspace{3mm}

\subsection*{Grupp nummer: ......}

\vspace{3mm}

\subsection*{Övningsledare: ..............................................................}


\vspace{10mm}

\begin{tabular}{l}
 \hspace{140mm} \\
\hline \hline
\end{tabular}

\vspace{5mm}

\subsection*{Betyg: ..... \hspace{2mm}  Datum: .............. \hspace{2mm} Rättad av: ........................................}


%\end{document}

\clearpage
\thispagestyle{empty}
\mbox{} % empty page for duplex
\clearpage 

% Huvudinlämning
\setcounter{page}{1}

\section{Vad är det här?}

Det här är en mall för att lämna in sina INDA-labbar på ett snyggt och enkelt
sätt, t.ex. för att man vill förtjäna guldstjärnor av sin asse. Mallen är
förhoppningsvis tillräckligt kommenterad för den ska gå att använda utan större
svårigheter.

Kom ihåg att byta ut ditt namn, kod och innehåll innan du lämnar in
inlämningen. :)

\section{Källkod}

På det här sättet kan man inkludera en hel fil, så att du inte behöver hålla på
och kopiera det in i \LaTeX. Notera att detta exempel förutsätter att nämnda
kod-fil ligger i samma mapp som denna .tex-fil. Annars får man nog skriva
relativa sökvägar (\texttt{path/to/file.java}), testa, men det borde ju gå bra?

\subsection{Foo.java}
\lstinputlisting{Foo.java} % Fil som ska läsas in
% Notera här att filen antas ligga i samma mapp som denna .tex-fil

\subsection{Inkludering av kod direkt i \LaTeX}

Ibland ska bara mindre kodsnuttar som skaredovisas, då kan det vara smidigare
att lägga koden direkt i detta .tex-dokument som i detta exempel.

För att skriva ut ``Hello, World!'' i Java så kan man lämpligtvis köra följande
metodanrop:

\begin{lstlisting}
System.out.println("Hello World!");
\end{lstlisting}

\end{document}

